%%%%%%%%%%%%%%%%%%%%%%%%%%%%%%%%%%%%%%%%%%%%%%%%%%%%%%%%%%%%%%%%%%%%%%%%%%%%%%%%
% AMS Beamer series / Bologna FC / Template
% Andrea Omicini
% Alma Mater Studiorum - Università di Bologna
% mailto:andrea.omicini@unibo.it
%%%%%%%%%%%%%%%%%%%%%%%%%%%%%%%%%%%%%%%%%%%%%%%%%%%%%%%%%%%%%%%%%%%%%%%%%%%%%%%%
%\documentclass[handout]{beamer}\mode<handout>{\usetheme{default}}
%
\documentclass[presentation, 10pt]{beamer}\mode<presentation>{\usetheme{metropolis}}
%\documentclass[handout]{beamer}\mode<handout>{\usetheme{AMSBolognaFC}}
%%%%%%%%%%%%%%%%%%%%%%%%%%%%%%%%%%%%%%%%%%%%%%%%%%%%%%%%%%%%%%%%%%%%%%%%%%%%%%%%

%\usepackage{lmodern}
\usefonttheme{professionalfonts}

\usepackage{arev}
\usepackage{multicol}
\usepackage{wasysym}
\usepackage{amsmath,blkarray}
\usepackage{centernot}
\usepackage{fontawesome}
\usepackage{fancyvrb}
\usepackage{minted}
\usepackage{tikz}
\usepackage[ddmmyyyy]{datetime}
\usepackage{xpatch}
\usepackage{tcolorbox}
\usepackage{hyperref}
%\xpretocmd{\inputminted}{\begin{tcolorbox}}{}{}%
%\xapptocmd{\inputminted}{\end{tcolorbox}}{}{}%

\renewcommand{\dateseparator}{}

%\renewcommand{\thefootnote}{\fnsymbol{footnote}}
\newcommand{\version}{1}
\usepackage{biblatex}
	\makeatletter
\addbibresource{biblio.bib}

%%%%%%%%%%%%%%%%%%%%%%%%%%%%%%%%%%%%%%%%%%%%%%%%%%%%%%%%%%%%%%%%%%%%%%%%%%%%%%%%
\title[It's all about effects]
{It's all about effects}
%
\subtitle[Effect systems in Functional Programming]
{Effect systems in Functional Programming}

%
\author[Aguzzi]
{Gianluca Aguzzi, \href{mailto:gianluca.aguzzi@unibo.it}{gianluca.aguzzi@unibo.it}}
%
\institute[DISI, Univ.\ Bologna]
{Dipartimento di Informatica -- Scienza e Ingegneria (DISI)\\\textsc{Alma Mater Studiorum} -- Universit{\`a} di Bologna}
%
\setminted[scala]{fontsize=\footnotesize, tabsize=2}
\renewcommand{\dateseparator}{/}
\date[\today]{\today \\
\url{https://github.com/cric96/asmd-side-effect-fp}
}
%
%%%%%%%%%%%%%%%%%%%%%%%%%%%%%%%%%%%%%%%%%%%%%%%%%%%%%%%%%%%%%%%%%%%%%%%%%%%%%%%%
\begin{document}

\nocite{*}
%%%%%%%%%%%%%%%%%%%%%%%%%%%%%%%%%%%%%%%%%%%%%%%%%%%%%%%%%%%%%%%%%%%%%%%%%%%%%%%%

%/////////
\frame{\titlepage}
%/////////
%% print toc
%\begin{frame}{Outline}
%	\tableofcontents
%\end{frame}

\begin{frame}{Acknowledgments}
	This presentation is mainly based on the work of \emph{Giacomo Cavalieri} that I would like to thank for his work and his support.
	
	Any error or misunderstanding is my fault.

	Thesis: \url{https://github.com/giacomocavalieri/master-thesis}
\end{frame}
\section{Effect Systems}
\begin{frame}{Side effects}
\begin{alertblock}{Definition}
 A \emph{side effect} is any application state change that is observable outside the called function other than its return value.
\end{alertblock}
\begin{exampleblock}{Examples}
	\begin{itemize}
	\item I/O operations
	\item Exceptions
	\item Global State changes
	\item Non-determinism
	\end{itemize}
\end{exampleblock}
\end{frame}
\begin{frame}[standout]
\begin{center}
\huge{Side effects are lies. }

\large{Your function promises to do \emph{one} thing, but it also does other \emph{hidden} things.}
\end{center}
\end{frame}
\begin{frame}{Side effects -- Cont.}
\begin{exampleblock}{Why we should care about side effects?}
	\begin{itemize}
	\item \emph{Reasoning}: side effects make it harder to reason about the code
	\item \emph{Testing}: side effects make it harder to test the code
	\item \emph{Concurrency}: side effects make it harder to reason about concurrency
	\item \emph{Refactoring}: side effects make it harder to refactor the code
	\end{itemize}
\end{exampleblock}
\begin{alertblock}{How?}
	\begin{itemize}
		\item Explicitly model side effects in the type system
		\item Leverage an \emph{effect system} to manage side effects
	\end{itemize}
\end{alertblock}
\end{frame}

\begin{frame}{Effect system}
\begin{exampleblock}{Definition}
An \emph{effect system} is a type system that tracks the \emph{effects} of a program.
\end{exampleblock}
\begin{exampleblock}{Examples}
	\begin{itemize}
	\item Checked exceptions in Java
	\item Monads (?)
	\item Scala 3.0 \texttt{Capabilities}? (Next :))
	\item Algebraic Effects
	\end{itemize}
\end{exampleblock}
\begin{alertblock}{Today Lesson}
	\begin{itemize}
		\item Focus on Monads 
		\item Briefly introduce Algebraic Effects
	\end{itemize}
\end{alertblock}
\end{frame}

\begin{frame}[fragile]{Side effects: I/O monad (recap)}
\begin{exampleblock}{I/O monad}
	\begin{itemize}
		\item The I/O monad is a way to model side effects in Scala
		\item It is a \emph{monad} that represents a sequence of I/O operations
		\item It is a \emph{pure} way to model side effects
	\end{itemize}
\end{exampleblock}
\inputminted[firstline=6,lastline=17]{scala}{code/src/main/scala/monads/IO.scala}
\end{frame}

\begin{frame}{IO usage}
\begin{itemize}
	\item \texttt{IO} can be used in a for-comprehension
	\item The for-comprehension is a way to manage side effects
	\begin{itemize}
		\item \emph{Programs} became first class citizens of the language
	\end{itemize}
	\item After composing the I/O operations, we can run the program (i.e., the end of the world)
\end{itemize}
\inputminted[firstline=42,lastline=48]{scala}{code/src/main/scala/monads/IO.scala}
\end{frame}

\begin{frame}{Composing effects}
\begin{itemize}
	\item IO is the way to model I/O operations
	\item What about other computational effects?
	\begin{itemize}
		\item Exceptions?
		\item State?
		\item Async computations?
	\end{itemize}
	\item We need a way to \textbf{compose} different effects
	\item \textbf{Remember}!! Everything is about \emph{composition} :)
	\item Several approaches:
	\begin{itemize}
		\item Monad Stacks
		\begin{itemize}
			\item Namely, create a new monad that is a composition of \emph{two or more} monads
		\end{itemize}
		\item Monad Transformers Library
		\begin{itemize}
			\item A comprehensive library that provides a way to compose different effects
			\item It is a way to reduce the boilerplate of monad stacks
		\end{itemize}
	\end{itemize}
\end{itemize}
\end{frame}
\begin{frame}[fragile]{Motivating Example -- Parser}
\begin{itemize}
	\item A parser is a program that takes a string and returns an Option of a parsed value and the remaining string
\end{itemize}
\inputminted[firstline=5, lastline=5]{scala}{code/src/main/scala/monads/Parser.scala}
\begin{itemize}
	\item A Parser can be modelled as a monad
\end{itemize}
\inputminted[firstline=7, lastline=17]{scala}{code/src/main/scala/monads/Parser.scala}

\end{frame}
\begin{frame}[fragile]{Motivating Example -- Parser}
\begin{itemize}
	\item A Parser is similar to a State monad
\end{itemize}
\begin{tcolorbox}
\begin{minted}[fontsize=\scriptsize]{scala}
final case class Parser[A](parse: String => Option[(A, String)]) 
final case class State[S, A](run: S => (A, S))
\end{minted}
\end{tcolorbox}
\begin{itemize}
	\item Particularly, is a composition of a State monad and an Option monad
	\item We can use a \emph{monad transformer} to compose the two monads
\end{itemize}
\end{frame}
%===============================================================================
\section{Monad Stacks}
\begin{frame}{Monad Transformers}
\begin{exampleblock}{Definition}
A \emph{monad transformer} is a \textbf{type constructor} that takes a monad as \emph{an argument} and returns a new monad that combines the effects of both monads.
\end{exampleblock}
\begin{alertblock}{Why}
	\begin{itemize}
		\item Composing multiple effect types in a \textbf{single computation}
		\item Avoiding the boilerplate of deeply nested monads
		\item Providing a unified interface for working with combined effects
	\end{itemize}
\end{alertblock}
\begin{exampleblock}{Common Examples}
	\begin{itemize}
		\item \texttt{OptionT[M, A]}: Adds optionality effects to monad \texttt{M}
		\item \texttt{StateT[S, M, A]}: Adds stateful computation to monad \texttt{M}
	\end{itemize}
\end{exampleblock}
\end{frame}
\begin{frame}[fragile]{Monad Transformers -- Cont.}
\begin{alertblock}{Formal Definition}
	A monad transformer is a tuple \texttt{(T, lift)} where:
	\begin{itemize}
		\item \texttt{T} is a type constructor that takes as input a type constructor \texttt{M}, a type \texttt{A}, and returns a type \texttt{T[M, A]}.
		\item If \texttt{M} is a monad, then \texttt{T[M, *]} is also a monad
		\item \texttt{lift} is a polymorphic function with type \texttt{M[A] => T[M, A]} that lifts a monadic value into the transformed monad
		\begin{itemize}
			\item \texttt{lift} is a way to inject a value of type \texttt{M[A]} into the transformed monad \texttt{T[M, A]}
		\end{itemize}
	\end{itemize}
\end{alertblock}
\begin{tcolorbox}
\begin{minted}{scala}
trait MonadTransformer[T[_[_], _]]:
  def lift[M[_]: Monad, A](m: M[A]): T[M, A]
\end{minted}
\end{tcolorbox}
\end{frame}
\begin{frame}[fragile]{Monad Transformers -- OptionT}
	\begin{center}
	\begin{tikzpicture}
    \node[circle, fill=black!5, draw, minimum size=3 cm,label=above:{\mintinline{scala}{M}}, label=below:{\mintinline{scala}{OptionT[M[_], _]}}] {};
    \node[circle, fill=black!20, draw, minimum size=1.5 cm,label=above:{\mintinline{scala}{Option}}] {};
  \end{tikzpicture}
	\end{center}
\begin{tcolorbox}
\begin{minted}{scala}
final case class OptionT[M[_], A](runOptionT: M[Option[A]])
given MonadTransformer[OptionT] with 
	def lift[M[_]: Monad, A](m: M[A]): OptionT[M, A] = 
		OptionT(m.map(Some(_)))
\end{minted}
\end{tcolorbox}
\end{frame}

\begin{frame}[fragile]{Monad Transformers -- OptionT}
\begin{itemize}
	\item Now it is possible to compose the \texttt{Option} effect with another monad
\end{itemize}
\begin{tcolorbox}
\begin{minted}{scala}
def fail[M[_]: Monad, A]: OptionT[M, A] = 
	OptionT(Monad.pure(None)) 

def failAndIO: OptionT[IO, Unit] = 
 for 
 		_ <- IO.putStrLn("Hello, world!").lift 
		_ <- OptionT.fail: OptionT[IO, Unit] 
		_ <- IO.putStrLn("Unreachable").lift 
	yield ()	

@main def main: Unit = failAndIO.runOptionT.unsafeRun()
\end{minted}
\end{tcolorbox}
\end{frame}
\begin{frame}[fragile]{Monad Transformers -- StateT}
\begin{center}	
\begin{tikzpicture}
	\node[circle, fill=black!5, draw, minimum size=3 cm,label=above:{\mintinline{scala}{M}}, label=below:{\mintinline{scala}{StateT[S, M[_], _]}}] {};
	\node[circle, fill=black!20, draw, minimum size=1.5 cm,label=above:{\mintinline{scala}{State}}] {};
\end{tikzpicture}
\begin{tcolorbox}
\begin{minted}{scala}
final case class StateT[S, M[_], A](
	runStateT: S => M[(A, S)]
)

type StateTFixS[S] = [M[_], A] =>> StateT[S, M, A]

given [S]: MonadTransformer[StateTFixS[S]] with
	def lift[M[_]: Monad, A](m: M[A]): StateT[S, M, A] =
		StateT(s => m.map((_, s)))
\end{minted}
\end{tcolorbox}
\end{center}
\end{frame}
\begin{frame}[fragile]{Monad Transformers -- StateT}
As with OptionT, we can compose the State effect with another monad
\begin{tcolorbox}
\begin{minted}{scala}
// Operations
def get[S, M[_]: Monad]: StateT[S, M[_], S] =
	StateT(s => Monad.pure((s, s)))

def set[S, M[_]: Monad](state: S): StateT[S, M[_], Unit] =
	StateT(_ => Monad.pure(((), state)))

// Example
def stateAndIO: StateT[String, IO, Unit] =
	for
		state <- StateT.get[String, IO]
		_     <- IO.putStrLn(f"Current state: $state").lift
		_     <- IO.putStrLn("Setting new state").lift
		_     <- StateT.set[String, IO]("bar")
	yield ()
\end{minted}
\end{tcolorbox}
\end{frame}
\begin{frame}[fragile]{Monad Stacks}
\begin{itemize}
	\item Monad transformers allow composing different effects
	\begin{itemize}
		\item They inject capabilities from one monad into another
		\item For example, adding IO effects to a State monad
	\end{itemize}
	\item Multiple transformers can be stacked together
	\begin{itemize}
		\item Creating what's known as a \emph{monad stack}
	\end{itemize}
\end{itemize}
\begin{tcolorbox}
\begin{minted}{scala}
type Program[A] = StateT[String, OptionT[IO, _], A]
\end{minted}
\end{tcolorbox}

\begin{itemize}
	\item This type represents programs that:
	\begin{itemize}
		\item Perform I/O operations (innermost layer - IO)
		\item Can potentially fail (middle layer - OptionT)
		\item Manipulate string state (outermost layer - StateT)
	\end{itemize}
\end{itemize}
\end{frame}

\begin{frame}[fragile]{Monad Transformers -- Order matter}
\begin{itemize}
	\item The order of monad transformers in a stack matters significantly
\end{itemize}
\begin{tcolorbox}
\begin{minted}{scala}
// Different stacking orders produce different semantics
type Program1[A] = StateT[String, OptionT[IO, _], A]
type Program2[A] = OptionT[StateT[String, IO, _], A]
\end{minted}
\end{tcolorbox}
\begin{itemize}
	\item In \mintinline{scala}{Program1}, state changes are lost if failure occurs
	\item In \mintinline{scala}{Program2}, state changes persist even when returning None
\end{itemize}
\begin{multicols}{2}
	\centering
	\resizebox*{0.25\textwidth}{!}{
	\begin{tikzpicture}
		\node[circle, fill=black!5, draw, minimum size=4.5 cm,label=above:{\mintinline{scala}{IO}}, label=below:{\mintinline{scala}{Program1}}] {};
		\node[circle, fill=black!20, draw, minimum size=3 cm,label=above:{\mintinline{scala}{Option}}] {};
		\node[circle, fill=black!30, draw, minimum size=1.5 cm,label=above:{\mintinline{scala}{State}}] {};
	\end{tikzpicture}
	}
	\centering
	\resizebox*{0.25\textwidth}{!}{
	\begin{tikzpicture}
		\node[circle, fill=black!5, draw, minimum size=4.5 cm,label=above:{\mintinline{scala}{IO}}, label=below:{\mintinline{scala}{Program2}}] {};
		\node[circle, fill=black!20, draw, minimum size=3 cm,label=above:{\mintinline{scala}{State}}] {};
		\node[circle, fill=black!30, draw, minimum size=1.5 cm,label=above:{\mintinline{scala}{Option}}] {};
	\end{tikzpicture}}
\end{multicols}
\end{frame}

\begin{frame}[fragile, shrink=15]{Monad Transformers -- Parser Rewritten}
\begin{tcolorbox}
\begin{minted}{scala}
// Reducing the boilerplate of nested monads
type State[S, A] = StateT[S, Identity, A]
def get[S]: State[S, S] = StateT.get
def set[S](state: S): State[S, Unit] = StateT.set(state)
// New parser definition
opaque type Parser[A] = OptionT[State[String, _], A]
extension [A](parser: Parser[A])
	def parse(string: String): (Option[A], String) =
		parser.runOptionT.runState(string)
// Operations on the parser (API)
def fail[A]: Parser[A] = OptionT.fail
def get: Parser[String] = State.get.lift[OptionT]
def set(string: String): Parser[Unit] =
	State.set(string).lift[OptionT]
// Parsing example
def char: Parser[Char] =
	for
		string <- get
		result <- string.headOption match
			case Some(char) =>
				for _ <- set(string.tail)
				yield char
			case None => fail
	yield result
\end{minted}
\end{tcolorbox}
\end{frame}
\begin{frame}[fragile,shrink=35]{Monad Transformers -- Limits}
\begin{exampleblock}{Manual Lifting of Operations}
	\begin{itemize}
		\item The necessity for \emph{manual lifting operations} when using monad transformers \textbf{complicates code}.
  \item Leads to \textbf{significant boilerplate}, overshadowing the application's \emph{actual logic}.
  \item Code alteration for stack changes demands \textbf{extensive rewriting}.
	\end{itemize}
\begin{tcolorbox}
\begin{minted}{scala}
def manualLifting: OptionT[IO, String] =
	for
		_ <- IO.putStrLn("Hello, world!").lift
		_ <- OptionT.fail: OptionT[IO, Unit]
	yield ()
\end{minted}
\end{tcolorbox}
\end{exampleblock}
\begin{exampleblock}{Principle of Least Privilege}
	\begin{itemize}
		\item Fixing the monad stack violates the \textbf{principle of least privilege}.
		\item Forces \emph{unnecessary capabilities} onto parts of the application.
	\end{itemize}
\begin{tcolorbox}
\begin{minted}{scala}
type App = OptionT[IO, _] // "-Ykind-projector:underscores"
def main: App = for
	rawData <- readFile("data.csv") // App
	csv 		<- parseCsv(rawData) // App
	column  <- csv.get[Int]("column") // App
yield column.sum
\end{minted}
\end{tcolorbox}
\end{exampleblock}
\begin{exampleblock}{Encapsulation Violation}
\begin{itemize}
	\item Monad transformers \textbf{tightly couple} code to specific effect modeling.
	\item Ties logic to a \emph{specific implementation}, severely \textbf{hindering future changes}.
\end{itemize}
\end{exampleblock}
\end{frame}
%===============================================================================
\section{Monad Transformers Library}
\begin{frame}{Monad Transformers Library}
	\begin{exampleblock}{What}
		\begin{itemize}
			\item MTL \emph{simplifies} side effect management in functional programming by \textbf{abstractly describing} side effects and allowing for \emph{flexible interpretation}.
			\item State-of-the-art scala library: \texttt{cats-mtl} \url{https://typelevel.org/cats-mtl/}
		\end{itemize}
	\end{exampleblock}
	\begin{exampleblock}{How}
		\begin{itemize}
			\item MTL leverages \textbf{type classes} to define \emph{families} of effects, each characterized by operations that express possible actions.
		  \item This approach facilitates \textbf{modular}, \textbf{composable} \emph{side-effect} handling, \emph{significantly} reducing boilerplate and \textbf{improving code maintainability}.
		\end{itemize}
	\end{exampleblock}	
\end{frame}


\begin{frame}[fragile]{MTL -- Basic example}
\begin{tcolorbox}
\begin{minted}{scala}
// M is a type constructor that represents the effect
trait State[M[_], S]:
	def get: M[S]
	def set(s: S): M[Unit]

// Useful for context bound
type HasState[S] = [M[_]] =>> State[M, S]

// Given a M 
def update[S, M[_]: Monad: HasState[S]](f: S => S): M[Unit] =
	for
		s <- summon[State[M, S]].get
		_ <- summon[State[M, S]].set(f(s))
	yield ()
\end{minted}
\end{tcolorbox}
\end{frame}
\begin{frame}[fragile]
	\frametitle{Composition of Multiple Effects}
	
	\begin{itemize}
			\item MTL \textbf{excels} in combining \emph{different side effects} with ease, removing the \textbf{rigidity} observed with traditional monad transformers.
	\end{itemize}
\begin{tcolorbox}
	
\begin{minted}{scala}
	def effects[M[_]: HasState[Int]: Fail: Monad]: M[String] =
	for
		state    <- State[M, Int].get
		newState <- divide(10, state)
		_        <- State[M, Int].set(newState)
	yield f"The result is $newState"

// decoupling the effect from the implementation
effects[StateT[Int, IO, _]].runStateT(10).unsafeRun()
\end{minted}
\end{tcolorbox}

\begin{itemize}
	\item \mintinline{scala}{M[_]} can be read as: ``given a monad M that has the effect of State and can Fail.''
\end{itemize}	
\end{frame}
\begin{frame}[fragile]
	\frametitle{MTL as an \textbf{Effect System}}
		
		\begin{itemize}
			\item MTL can be viewed as an \emph{early example} of an effect system, \textbf{formally describing} the side effects of computations.
			\item It supports defining \emph{arbitrary effect classes} at the granularity most suited to the problem, offering a \textbf{powerful design tool} for effect-oriented systems.
		\end{itemize}
\begin{tcolorbox}
\begin{minted}{scala}
// Effect definition
trait UserStore[M[_]]:
	def get(userId: UserId): M[Option[User]]
	def save(user: User): M[Unit]
	def delete(userId: UserId): M[Unit]

// Operations

def updateOrDelete[M[_]: UserStore]
 	(user: User)(f: User =>  User | Delete): M[Unit]

\end{minted}
\end{tcolorbox}
\end{frame}
\begin{frame}[fragile]{MTL as an Effect System}
	\begin{itemize}
		\item We can inteprete the \mintinline{scala}{UserStore} effect in different ways
		\begin{itemize}
			\item In memory \faArrowRight \, testing
			\item Database \faArrowRight \, production
		\end{itemize}
		\item This is important for testing and for the separation of concerns
		\begin{itemize}
			\item Easy to test and to reason about the code
		\end{itemize}
	\end{itemize}
\begin{tcolorbox}
\begin{minted}{scala}
// Production
final case class DatabaseConnection()
final case class Runtime(connection: DatabaseConnection)

type ProductionRunner = StateT[Runtime, IO, _]

// Testing
final case class Runtime(users: Map[UserId, User])
type TestRunner = State[Runtime, _]
\end{minted}
\end{tcolorbox}
\end{frame}
%===============================================================================
\section{Algebraic Effects}
\begin{frame}{Understanding Algebraic Effects}
	\begin{exampleblock}{What are Algebraic Effects?}
		\begin{itemize}
			\item A formalism to model side effects in pure functional programming
			\item Provide a separation between the definition of effects and their implementation
			\item Allow for compositional and modular effect handling
		\end{itemize}
	\end{exampleblock}

	\begin{alertblock}{Key Concepts}
		\begin{itemize}
			\item \textbf{Effects and Handlers:} Effects are operations that represent side effects, while handlers define their implementation
			\item \textbf{Composability:} Effects can be combined, allowing complex operations to be built from simpler ones
			\item \textbf{Purity:} Keeps the core logic pure by abstracting effects, making code easier to reason about and test
		\end{itemize}
	\end{alertblock}
\end{frame}
\begin{frame}{Algebraic Effects -- Langauge and Libraries}
\begin{itemize}
	\item \textbf{koka}: a functional language with effect handlers by desing \url{https://koka-lang.github.io/koka/doc/index.html}
	\item \textbf{unison}: a modern approach to distributed programming that use abilities at the type level
	\item \textbf{Effekt}: a library for Scala 3.0 that provides a way to model algebraic effects \url{https://b-studios.de/scala-effekt/}
\end{itemize}
\end{frame}
\begin{frame}[fragile, shrink=40]{Algebraic Effects -- Effekt}
	\begin{itemize}
		\item Definie an effect
	\end{itemize}
\begin{tcolorbox}
\begin{minted}{scala}
trait Flip {
	def flip: Control[Boolean]
}
\end{minted}
\end{tcolorbox}
\begin{itemize}
	\item Use the effect
\end{itemize}
\begin{tcolorbox}
\begin{minted}{scala}
def program(flippable: Flip): Control[Int] = for {
	x <- flippable.flip()
} yield if (x) 2 else 3
\end{minted}
\end{tcolorbox}
\begin{itemize}
	\item Define the handler
\end{itemize}
\begin{tcolorbox}
\begin{minted}{scala}
def ambList[R](prog: Flip => Control[R]): Control[List[R]] =
  new Handler[List[R]] with Flip {
    def flip() = use { 
				resume => for {
      		ts <- resume(true)
      		fs <- resume(false)
    		} yield ts ++ fs 
		}
  } handle { amb => prog(amb) map { r => List(r) } }
\end{minted}
\end{tcolorbox}
\begin{itemize}
	\item We will see that Scala 3.0 Capabilities are similar to algebraic effects :)
\end{itemize}
\end{frame}
\section{Conclusion}
\begin{frame}{Conclusion}
	\begin{alertblock}{Key Takeaways}
		\begin{itemize}
			\item Effect systems provide systematic management of side effects in functional programming
			\item Monad transformers enable composition of different effectful computations
			\item MTL abstracts effect handling through type classes, improving modularity and testability
			\item Algebraic effects offer a powerful formalism separating effect definition from implementation
		\end{itemize}
	\end{alertblock}
	
	\begin{exampleblock}{Coming Next}
		\begin{itemize}
		\item 
		\textbf{Asynchronous Computations with Direct Style and Introduction to Capabilities}
		\end{itemize}
	\end{exampleblock}
\end{frame}
%/////////
\frame{\titlepage}
%/////////


%%%%
\setbeamertemplate{page number in head/foot}{}
%/////////

%/////////

%%%%%%%%%%%%%%%%%%%%%%%%%%%%%%%%%%%%%%%%%%%%%%%%%%%%%%%%%%%%%%%%%%%%%%%%%%%%%%%%
\end{document}
%%%%%%%%%%%%%%%%%%%%%%%%%%%%%%%%%%%%%%%%%%%%%%%%%%%%%%%%%%%%%%%%%%%%%%%%%%%%%%%%
